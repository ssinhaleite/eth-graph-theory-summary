\documentclass[11pt,article,oneside,a4paper]{memoir}

%% Packages
%% ========

%% many common packages
\input{commonpackages}

%% Some more packages that you may want to use.  Have a look at the
%% file, and consult the package docs for each.
\input{extrapackages}

%% Our layout configuration.
\input{layoutsetup}

%% Theorem environments.  You will have to adapt this for a German
%% thesis.
\input{theoremsetup}

%% Helpful macros.
\input{macrosetup}

%%page layout settings and listing templates etc.
\input{settings}

\title{Graph Theory, Spring Semester 2017}
\author{
Vanessa Leite\\
\vspace{2em}
Github (git/svn) repository page:\\ \url{https://github.com/ssinhaleite/eth-graph-theory-summary}\\
Contact \href{mailto:vrcleite@gmail.com}{vrcleite@gmail.com} if you have any questions.}
\thesistype{Summary of the lectures in 2017}
\advisors{Professor:\\ Prof.\ Dr.\ Benjamin Sudakov}
\department{D-MATH, ETH}
\date{\today}

\begin{document}
\frontmatter

%% Title page is auto-generated from document information above.
%% DO NOT CHANGE.
\begin{titlingpage}
  \calccentering{\unitlength}
  \begin{adjustwidth*}{\unitlength-24pt}{-\unitlength-24pt}
    \maketitle
  \end{adjustwidth*}
\end{titlingpage}

\mainmatter
\newpage
\chapterprecishere{
Summary to summarize the content of the lecture about Graph Theory of Prof. Dr. Benjamin Sudakov. You can find a referece list in the end of each chapter.\\---}
\newpage

%% This change is needed if the article option for the memoir document class
%% is used, in order to count sections (article) as if they were chapters (memoir)
\counterwithout{section}{chapter}

%% Our content

\newpage
\clearpage
\pagenumbering{roman}
\setcounter{tocdepth}{3}
\setcounter{secnumdepth}{2}
\tableofcontents

\clearpage
\pagenumbering{arabic}

\newpage

\subfile{01-basic-notions.tex}

\subfile{02-trees.tex}

\subfile{03-vertex-edges-connectivity.tex}

\subfile{04-eulerian-graphs.tex}

\subfile{05-matchings.tex}

\subfile{06-planar-graphs.tex}

\subfile{07-graph-coloring.tex}

\pagebreak

\subfile{08-Glossary.tex}

\subfile{09-TODO.tex}

\end{document}
